\documentclass[a4paper]{article}
\usepackage[utf8]{ctex}
\usepackage{graphicx}

\title{作业三:AVL Tree}
\author{李沁霞 \\ 统计学 3210300363}
\date{\today}

\begin{document}

\maketitle

\section{项目简介}
AVL Tree为BST的平衡树,平衡因素有左右子树之间的差距影响。若差距不超过1是平衡树,反之依然。可以使用转换解决树的平衡性。转换程序设计可以在AvlTree.h的heading文件寻找。

\section{设计思路}
设计vector函数的列表,输入元素为[k1,k2]之间的树,之后测试程序的运行时间。根据树的高度有三种不同的时间复杂形势。我们将三种时间复杂性测试程序的运行时间。

\section{理论分析}
由于树的不同高度情况,导致程序有不同的时间复杂性。以下是根据AVL Tree的高度而产生的间复杂性:
\begin{enumerate}
    \item 最佳案例:当$k <= log(N)$时,程序的时间复杂性为$O(log(N))$.
    \item 最差案例:当$k = N$时,程序的时间复杂性为$O(N)$.
    \item 平均案例:当$k = log_{n}(N)$时,程序的时间复杂性为$O(k + log(N))$.
\end{enumerate}

\section{数据结果分析}
\begin{itemize}
\item 测试结果
\begin{verbatim}
当 k <= log(N):
-1 829 -200 500
-200 -1 500 829
当 k = N : 
The run time of 2000    in test 1 cost: 0.002016 s.
The run time of 20000   in test 1 cost: 7e-06 s.
The run time of 200000  in test 1 cost: 8e-06 s.
The run time of 2000000 in test 1 cost: 3.1e-05 s. 
当 k = logn(N) :
The run time of 65536   in test 2 cost: 0.000437 s.
The run time of 262144  in test 2 cost: 1e-05 s.
The run time of 1048576 in test 2 cost: 1.3-05 s.
The run time of 4194304 in test 2 cost: 1.4e-05 s.
\end{verbatim}
\end{itemize}
\begin{center}
\begin{tabular}{|c|c|c|c|c|c|c|c|}
    \hline
    \multicolumn{2}{|c}{$O(log(N))$} & \multicolumn{3}{|c|}{$O(N)$}  &  \multicolumn{3}{|c|}{$O(k + log(N))$} \\
    \hline
    insert & output & $k$ & $n$ & time(s) & $k1$ & $k2$ & time(s) \\
    \hline
    -1 & -200 & 2000 & 2000 & 0.002016 & $2^{16}$ & $log_2(2^{16})$ & 0.000437 \\
    \hline
    829 & -1 & 20000 & 20000 & 7e-06 & $2^{18}$ & $log_2(2^{18})$ & 1e-05 \\
    \hline
    -200 & 500 & 200000 & 200000 & 8e-06 & $2^{20}$ & $log_2(2^{20})$ & 1.3-05 \\
    \hline
    500 & 829 & 2000000 & 2000000 & 3.1e-05 & $2^{22}$ & $log_2(2^{22})$ & 1.4e-05 \\
    \hline
\end{tabular}
\end{center}
从表上可知,$O(log(N))$的运行时间是最优的,很快就可以输出列表中的元素。而$O(k + log(N))$的运行时间也好,虽然输入的元素比O(N)比较大,但是运行结果更快。这就说明$O(N)$的时间复杂度是最差的。

\section{结论}
通过头文件理解AvlTree的程序设计思路,另外设计了main.cpp文件测试AvlTree的时间复杂性运算。由头文件和main.cpp可知树的高度影响到程序的运行时间。

\end{document}
