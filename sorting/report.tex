\documentclass[a4paper]{article}
\usepackage[utf8]{ctex}
\usepackage{amsfonts}
\usepackage{amsmath}
\usepackage{multirow}

\title{\textbf{作业六:排序算法}}
\author{李沁霞 \\ 3210300363 统计学}
\date{\today}

\begin{document}

\maketitle

\section{简介}
我们比较两种强大的排序算法,快速排序(quicksort)和堆排序(heapsort)。快速排序通常在实践中使用,而内存有问题时使用堆排序。
\\
首先简单的介绍一下快速排序和堆排序的算法流程,然后比较它们的测试结果。

\section{快速排序}
快速排序基于分而治之的方法。总的来说,快速排序算法遵循三个主要步骤:
\begin{enumerate}
    \item 选取一个元素作为基准
    \item 将较小的元素移动到左侧,而大的元素移动到右侧
    \item 在每个分区上重复上述步骤
\end{enumerate}
平均而言,最佳情况下快速排序算法的时间复杂度为$O(nlogn)$。最坏情况的时间复杂度为$O(n^2)$,如果选错了主元就发生这种情况。快速排序实现时不稳定,则最佳情况下的时间复杂度可能为$O(logn)$,而平均情况为$O(n)$。

\section{堆排序}
堆排序是一种基于二叉堆数据结构的比较排序方法。堆排序算法有四个主要步骤:
\begin{enumerate}
    \item 构建一个最小(或最大)堆
    \item 此时,最小项存储在堆的根部。从根节点删除元素,并在右边的叶子储存在根节点中。
    \item 堆化树的根
    \item 当堆的大小大于1时,重复步骤2和3
\end{enumerate}
Percdown是一个以正确的顺序排列节点使他们遵循堆属性的过程。堆排序在所有情况下的时间复杂度都是$O(nlogn)$,但是堆排序使用$O(1)$辅助空间,所以如果内存有问题,使用堆排序是一个不错的选择。

\section{测试结果}
\begin{center}
\begin{tabular}{|c|c|c|c|c|}
    \hline
    Size & Efficiency \% & Heapsort (s) & Quicksort (s) \\
    \hline
    \multirow{4}{4em}{10000} & 1 & 0.003 & 0.001 \\
    & 10 & 0.005 & 0.001 \\
    & 90 & 0.004 & 0.001 \\
    & 99 & 0.004 & 0.001 \\
    \hline
    \multirow{4}{4em}{100000} & 1 & 0.059 & 0.012 \\
    & 10 & 0.062 & 0.007 \\
    & 90 & 0.046 & 0.006 \\
    & 99 & 0.045 & 0.01 \\
    \hline
    \multirow{4}{4em}{1000000} & 1 & 0.346 & 0.077\\
    & 10 & 0.333 & 0.08 \\
    & 90 & 0.318 & 0.063 \\
    & 99 & 0.307 & 0.067 \\
    \hline
\end{tabular}
\end{center}

\end{document}
