\documentclass[a4paper]{article}
\usepackage[utf8]{ctex}
\usepackage{amsfonts}
\usepackage{amsmath}
\usepackage{listings}

\title{\textbf{作业七:图的模板类设计}}
\author{李沁霞 \\ 3210300363 统计学}
\date{\today}

\begin{document}

\maketitle

\section{图形简介}
图是由顶点与边所组成的非线性数据结构。顶点是图的基本单位。有时,顶点也被称为节点。而边用于连接图形的两个节点,它又称为孤。图用于解决许多实现生活中的问题。图也用于社交网络,如LinkedIn、Facebook。

\section{设计思路}
\begin{tabular}{|c|c|}
    \hline
     Function &  Usage \\
     \hline
     void \texttt{choose\_store\_mode()} &  分别\texttt{adj\_mat}与\texttt{adj\_list}的列表 \\
     \hline
     void addEdge()  &  根据\texttt{store\_type}使用边连接两个节点 \\
     \hline
     void Init() &  根据\texttt{store\_type}增加函数 \\
     \hline
     void listVertex() & 打印Vertex中的值 \\
     \hline
     void listEdges() & 根据\texttt{store\_type}打印该字母与边 \\
     \hline
\end{tabular}

\section{测试结果}
\begin{lstlisting}
    adj_mat:
    A B C D
    (A,B,3)
    (A,C,2)
    (C,D,1)
    (D,B,5)
    (D,C,4)
    adj_list:
    A B C D
    (A,B,3)
    (A,C,2)
    (C,D,1)
    (D,B,5)
    (D,C,4)
    undirected graph:
    A B C D
    (A,B,1)
    (A,D,1)
    (B,C,1)
    (C,D,1)
\end{lstlisting}
邻接矩阵\texttt{(adj\_mat)}中的每个条目代表这些顶点之间的边的权重,而邻接表\texttt{(adj\_list)}中的一个指针数组指向连接该点的边。无向图(undirected graph)是连接在一起的一组对象,其中所有边都是双向。无向图有时被称为无向网络。

\end{document}
