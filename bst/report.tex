\documentclass[a4paper]{article}
\usepackage[UTF8]{ctex}
\usepackage{listings}

\title{作业二:BST排序算法实现}
\author{李沁霞\\ 统计学 3210300363}
\date{\today}

\begin{document}

\maketitle

\section{创建要求}
编写BSTSorting函数实现数组排序,而必须使用儿茶搜素树排序算法。编写两种不同的格式,一种是乱序后排序,另一种是不乱序排序。使用两种方法测试函数运行时间。

\section{设计思路}
\begin{enumerate}
    \item 编写BSTSorting函数的数组排序。
    \item 编写main函数的为头文件实现测试。
    \item 进行脚本文件编写。
\end{enumerate}

\section{添加函数}
\begin{lstlisting}
1.  void BSTSorting(vector<Comparable> &_arr, int _mode = 0)
    {
        BinarySearchTree<Comparable> tree;
        clock_t start,end;
        double time = 0;
        if(mode == 0)
        {
            start = clock;
            for(int i = 0; i < _arr.size(); ++i)
            {
                tree.insert(_arr[i]);
            }
            end = clock;
            time = double(end-start)/CLOCKS_PER_SEC;
        }
        if(mode == 1)
        {
            for(int j = 0; j < 100; ++j)
            {
                start = clock;
                tree.makeEmpty();
                for(int i< _arr.size()-1; i >= 1; --i)
                {
                    int k = random() % i;
                    Comparable temp = _arr[k];
                    _arr[k] = _arr[i];
                    _arr[i] = temp;
                }
                for(int i = 0; i < _arr.size(); ++i)
                {
                    tree.insert(_arr[i]);
                }
                end = clock;
                time += double(end-start)/CLOCKS_PER_SEC;
            }
        }
        if(_arr.size() <= 10000)
        {
            cout << "After sorting: " << endl;
            tree.printTree();
        }
        cout << "Time for executing: " << time << "s" << endl;
    }

2. int main(){
        int mode;
        vector <int> _arr;
        
        cout << "Enter sorting mode: ";
        cin >> mode;
        
        while(cin.fail() || (mode != 0 && mode != 1))
        {
            cin.clear();
            cout << "Please insert 0 or 1." << "Enter sorting mode: " << endl;
            cin >> mode;
        }
        cout << "Enter your array: " << endl;
        for(int temp = 0; cin >> temp;)
        {
            _arr.push_back(temp);
            if(cin.get() == '\n') break;
        }
        if(_arr.size() == 1)
        {
            int length = _arr.back();
            _arr.pop_back();
            for(int temp = length; temp >= 1; --temp)
            {
                _arr.push_back(temp);
            }
        }
        BSTSorting(_arr,mode);
        return 0;
    }
\end{lstlisting}

\section{Makefile与run脚本}
\begin{lstlisting}
1. make:
        g++ -o test main.cpp
    
   report:
        xelatex report.tex
        rm report.aux
        rm report.log
    
    .PHONNY:report
        
使用make:输入g++ -o test main.cpp测试程序的运行流程,
使用report:生成report.pdf的文件与remove某些report的文件。

2. #! /bin/bash

    make test
    make report
    ./test
    
使用make:执行test文件,使用./test命令测试main.cpp的程序
    
\end{lstlisting}

\section{测试说明}
对每一个数组都测试2种的时间复杂性模式.\\
Enter sorting mode: 0 \\
Enter your array: \\
10000 \\
Time for executing: 4.49529s \\

Enter sorting mode: 1 \\
Enter your array: \\ 
10000 \\
Time for executing: 0.238379s \\

Enter sorting mode: 0 \\
Enter your array: \\
50000 \\
Time for executing: 9.54646s \\

Enter sorting mode: 1 \\
Enter your array: \\
50000 \\
Time for executing: 1.73558s \\

上述测试结果显示乱序程序后排序(mode = 1)测试程序的时间比不乱序程序(mode = 0)更快. \\

使用Valgrind检查数组的泄露.\\
==5243== \\
==5243== HEAP SUMMARY: \\
==5243==     in use at exit: 133,311 bytes in 770 blocks \\
==5243==   total heap usage: 1,063 allocs, 293 frees, 158,123 bytes allocated \\
==5243== \\
==5243== LEAK SUMMARY: \\
==5243==    definitely lost: 12 bytes in 1 blocks \\
==5243==    indirectly lost: 0 bytes in 0 blocks \\
==5243==      possibly lost: 0 bytes in 0 blocks \\
==5243==    still reachable: 133,299 bytes in 769 blocks \\
==5243==         suppressed: 0 bytes in 0 blocks \\
==5243== Rerun with --leak-check=full to see details of leaked memory \\
==5243== \\
==5243== For lists of detected and suppressed errors, rerun with: -s \\
==5243== ERROR SUMMARY: 0 errors from 0 contexts (suppressed: 0 from 0) \\
    
\end{document}
